\documentclass[12pt,a4paper,twoside,openright]{report}
\let\openright=\cleardoublepage

% \usepackage{geometry}
% \geometry{bindingoffset=1cm}



%%% Choose a language %%%

\newif\ifEN
\ENtrue   % uncomment this for english
%\ENfalse   % uncomment this for czech

%%% Configuration of the title page %%%

\def\ThesisTitleStyle{mff} % MFF style
%\def\ThesisTitleStyle{cuni} % uncomment for old-style with cuni.cz logo
%\def\ThesisTitleStyle{natur} % uncomment for nature faculty logo

\def\UKFaculty{Faculty of Mathematics and Physics}
%\def\UKFaculty{Faculty of Science}

\def\UKName{Charles University in Prague} % this is not used in the "mff" style

% Thesis type names, as used in several places in the title
\def\ThesisTypeTitle{\ifEN BACHELOR THESIS \else BAKALÁŘSKÁ PRÁCE \fi}
%\def\ThesisTypeTitle{\ifEN MASTER THESIS \else DIPLOMOVÁ PRÁCE \fi}
%\def\ThesisTypeTitle{\ifEN RIGOROUS THESIS \else RIGORÓZNÍ PRÁCE \fi}
%\def\ThesisTypeTitle{\ifEN DOCTORAL THESIS \else DISERTAČNÍ PRÁCE \fi}
\def\ThesisGenitive{\ifEN bachelor \else bakalářské \fi}
%\def\ThesisGenitive{\ifEN master \else diplomové \fi}
%\def\ThesisGenitive{\ifEN rigorous \else rigorózní \fi}
%\def\ThesisGenitive{\ifEN doctoral \else disertační \fi}
\def\ThesisAccusative{\ifEN bachelor \else bakalářskou \fi}
%\def\ThesisAccusative{\ifEN master \else diplomovou \fi}
%\def\ThesisAccusative{\ifEN rigorous \else rigorózní \fi}
%\def\ThesisAccusative{\ifEN doctoral \else disertační \fi}



%%% Fill in your details %%%

% (Note: \xxx is a "ToDo label" which makes the unfilled visible. Remove it.)
\def\ThesisTitle{Climbing route editor}
\def\ThesisAuthor{Tomáš Sláma}
\def\YearSubmitted{2022}

% department assigned to the thesis
\def\Department{Department of Applied Mathematics}
% Is it a department (katedra), or an institute (ústav)?
\def\DeptType{Department}

\def\Supervisor{Martin Mareš}
\def\SupervisorsDepartment{Department of Applied Mathematics}

% Study programme and specialization
\def\StudyProgramme{Computer Science}
\def\StudyBranch{IPP1}

\def\Dedication{%
I'm grateful for the support from my friends and family, namely Kateřina Sulková, Jakub Medek, Matěj Kripner, Jan Černý, Benjamin Schiller and Jakub Pelc.
I would also like to thank my supervisor Martin Mareš for patience and valuable feedback.
}

\def\AbstractEN{%
With the rising popularity of sport climbing and bouldering, there are increasing incentives to take it to the online space.
This thesis does so in way suitable for regular climbing gyms.
A workflow for creating 3D hold and wall models and the accompanying editor has been developed, so that routes can be copied from the real world to the virtual one and be interacted with using software in a more immersive way.
}

\def\AbstractCS{%
S rostoucí popularitou sportovního lezení a boulderingu roste i pobídka po přenesení sportu do online prostoru.
Tato práce tak činí způsobem přístupným pro běžná lezecká centra.
Byl vyvinut způsob tvorby 3D modelů chytů a stěn a doprovázející 3D editor, aby cesty z reálného světa mohly být přeneseny do toho virtuálního k interaktivnímu prohlížení pomocí softwaru.
}

% 3 to 5 keywords (recommended), each enclosed in curly braces.
% Keywords are useful for indexing and searching for the theses by topic.
\def\Keywords{%
photogrammetry, routesetting, 3D modelling
}

% If your abstracts are long and do not fit in the infopage, you can make the
% fonts a bit smaller by this setting. (Also, you should try to compress your abstract more.)
% Alternatively, consider increasing the size of the page by uncommenting the
% geometry modification in thesis.tex.
\def\InfoPageFont{}
%\def\InfoPageFont{\small}  %uncomment to decrease font size

\ifEN\relax\else
% If you are writing a czech thesis, you additionally need to fill in the
% english translation of the metadata here!
\def\ThesisTitleEN{\xxx{Thesis title in English}}
\def\DepartmentEN{\xxx{Name of the department in English}}
\def\DeptTypeEN{\xxx{Department}}
\def\SupervisorsDepartmentEN{\xxx{Superdepartment}}
\def\StudyProgrammeEN{\xxx{study programme}}
\def\StudyBranchEN{\xxx{study branch}}
\def\KeywordsEN{%
\xxx{{key} {words}}
}
\fi


\newcommand{\ignore}[1]{}

% hack for properly dividing TOC (developer documentation was alone)
\usepackage{tocloft}
\setlength\cftparskip{0.06em}

\usepackage[a-2u]{pdfx}

\ifEN\else\usepackage[czech,shorthands=off]{babel}\fi
\usepackage[utf8]{inputenc}
\usepackage[T1]{fontenc}

% See https://en.wikipedia.org/wiki/Canons_of_page_construction before
% modifying the size of printable area. LaTeX defaults are great.
% If you feel it would help anything, you can enlarge the printable area a bit:
%\usepackage[textwidth=390pt,textheight=630pt]{geometry}
% The official recommendation expands the area quite a bit (looks pretty harsh):
%\usepackage[textwidth=145mm,textheight=247mm]{geometry}

%%% FONTS %%%
\usepackage{lmodern} % TeX "original" (this sets up the latin mono)

% Optionally choose an override for the main font for typesetting
\usepackage[mono=false]{libertinus} % popular for comp-sci (ACM uses this)
%\usepackage{tgschola} % Schoolbook-like (gives a bit of historic feel)
%\usepackage[scale=0.96]{tgpagella} % Palladio-like (popular in formal logic).

% Optionally choose a custom sans-serif fonts (e.g. for figures and tables).
% Default sans-serif font is usually Latin Modern Sans. Some font packages
% (e.g. libertinus) replace that with a better matching sans-serif font.
%\usepackage{tgheros} % recommended and very readable (Helvetica-like)
%\usepackage{FiraSans} % looks great
% DO NOT typeset the main text in sans-serif font!
% The serifs make the text easily readable on the paper.

% IMPORTANT FONT NOTE: Some fonts require additional PDF/A conversion using
% the pdfa.sh script. These currently include only 'tgpagella'; but various
% other fonts from the texlive distribution need that too (mainly the Droid
% font family).


% some useful packages
\usepackage{microtype}
\usepackage{amsmath,amsfonts,amsthm,bm}
\usepackage{graphicx}
\usepackage{xcolor}
\usepackage{booktabs}
\usepackage{caption}
\usepackage{floatrow}

\usepackage{minted}

\setminted{
frame=lines,
autogobble,
baselinestretch=1,
frame=lines,
framesep=6pt,
linenos
}

\usepackage{subfig}

\usepackage{svg}

\usepackage{siunitx}
\sisetup{mode=text}

% load bibliography tools
\usepackage[backend=bibtex,natbib,style=numeric,sorting=none]{biblatex}

\DeclareFieldFormat{url}{Available on-line at: \url{#1}}

\setcounter{biburllcpenalty}{9000}

% ----------------
\usepackage[edges]{forest}
\definecolor{folderbg}{RGB}{124,166,198}
\definecolor{folderborder}{RGB}{110,144,169}
\newlength\Size
\setlength\Size{4pt}
\tikzset{%
  folder/.pic={%
    \filldraw [draw=folderborder, top color=folderbg!50, bottom color=folderbg] (-1.05*\Size,0.2\Size+5pt) rectangle ++(.75*\Size,-0.2\Size-5pt);
    \filldraw [draw=folderborder, top color=folderbg!50, bottom color=folderbg] (-1.15*\Size,-\Size) rectangle (1.15*\Size,\Size);
  },
  file/.pic={%
    \filldraw [draw=folderborder, top color=folderbg!5, bottom color=folderbg!10] (-\Size,.4*\Size+5pt) coordinate (a) |- (\Size,-1.2*\Size) coordinate (b) -- ++(0,1.6*\Size) coordinate (c) -- ++(-5pt,5pt) coordinate (d) -- cycle (d) |- (c) ;
  },
}
\forestset{%
  declare autowrapped toks={pic me}{},
  pic dir tree/.style={%
    for tree={%
      folder,
      font=\ttfamily,
      grow'=0,
    },
    before typesetting nodes={%
      for tree={%
        edge label+/.option={pic me},
      },
    },
  },
  pic me set/.code n args=2{%
    \forestset{%
      #1/.style={%
        inner xsep=2\Size,
        pic me={pic {#2}},
      }
    }
  },
  pic me set={directory}{folder},
  pic me set={file}{file},
}
% -------------

% alternative with alphanumeric citations (more informative than numbers):
%\usepackage[backend=bibtex,natbib,style=alphabetic]{biblatex}
%
% alternatives that conform to iso690
% (iso690 is not formally required on MFF, but may help elsewhere):
%\usepackage[backend=bibtex,natbib,style=iso-numeric,sorting=none]{biblatex}
%\usepackage[backend=bibtex,natbib,style=iso-alphabetic]{biblatex}
%
% additional option choices:
%  - add `giveninits=true` to typeset "E. A. Poe" instead of full Edgar Allan
%  - `terseinits=true` additionaly shortens it to nature-like "Poe EA"
%  - add `maxnames=10` to limit (or loosen) the maximum number of authors in
%    bibliography entry before shortening to `et al.` (useful when referring to
%    book collections that may have hundreds of authors)
%  - for additional flexibility (e.g. multiple reference sections, etc.),
%    remove `backend=bibtex` and compile with `biber` instead of `bibtex` (see
%    Makefile)
%  - `sorting=none` causes the bibliography list to be ordered by the order of
%    citation as they appear in the text, which is usually the desired behavior
%    with numeric citations. Additionally you can use a style like
%    `numeric-comp` that compresses the long lists of citations such as
%    [1,2,3,4,5,6,7,8] to simpler [1--8]. This is especially useful if you plan
%    to add tremendous amounts of citations, as usual in life sciences and
%    bioinformatics.
%  - if you don't like the "In:" appearing in the bibliography, use the
%    extended style (`ext-numeric` or `ext-alphabetic`), and add option
%    `articlein=false`.
%
% possibly reverse the names of the authors with the default styles:
%\DeclareNameAlias{default}{family-given}

% load the file with bibliography entries
\addbibresource{refs}

% remove this if you won't use fancy verbatim environments
\usepackage{fancyvrb}

\hypersetup{unicode}
\hypersetup{breaklinks=true}

\usepackage[noabbrev]{cleveref}

\input{macros} % use this file for various custom definitions


\begin{document}

\include{title}

\tableofcontents


\chapwithtoc{Introduction}

With the rising popularity of climbing, in no small part due to the addition of the sport to the Tokyo 2020 Summer Olympics \cite{olympics}, climbing gyms are seeing a steady increase in new climbers.
An obvious attraction to both climbers and route setters is being able to model and display the current way the holds are set up (the current „setting“), offering a great number of advantages, such as

\begin{itemize}
	\item archiving older settings for tracking trends like difficulty and style
	\item being able to view the current setting online before visiting the gym and seeing if they are suitable (and possibly opting to go somewhere else if not)
	\item filtering boulders by difficulty to find suitable ones
	\item setting community-made boulders (if a suitable editor exists)
	\item adding a social aspect by (dis)liking and commenting on boulders, adding send videos and beta hints, etc.
\end{itemize}

Models of boulders are gradually becoming used in competitive bouldering and certain gyms, lead mainly by the OnlineObservation team \cite{onlineobservation}.
Their approach is simple -- take photos of the wall with the holds already on it and use them to generate a 3D model.
This works really well for a one-time model generation, but becomes infeasible for a bouldering gym that replaces boulders periodically, as the model would have to be regenerated each time, which takes a significant amount of time and specialized equipment.
It is also difficult to individually highlight certain boulders and edit them if a change is made after the setting.

This thesis attempts to solve this problem by focusing on what actually changes from setting to setting -- the position of the holds on the wall.
The repeated scanning of the holds and the wall adds redundancy, which could be removed by a program to edit the placements of the holds.
This adds time initially, since models of the wall and the holds have to be created.
However, it saves it in the long term and poses a number of advantages, making it a viable option for comercial bouldering gyms.


TODO: results

TODO: chapter overview

%Introduction should answer the following questions, ideally in this order:
%\begin{enumerate}
%\item What is the nature of the problem the thesis is addressing?
%\item What is the common approach for solving that problem now?
%\item How this thesis approaches the problem?
%\item What are the results? Did something improve?
%\item What can the reader expect in the individual chapters of the thesis?
%\end{enumerate}
%
%Expected length of the introduction is between 1--4 pages. Longer introductions may require sub-sectioning with appropriate headings --- use \texttt{\textbackslash{}section*} to avoid numbering (with section names like `Motivation' and `Related work'), but try to avoid lengthy discussion of anything specific. Any ``real science'' (definitions, theorems, methods, data) should go into other chapters.
%\todo{You may notice that this paragraph briefly shows different ``types'' of `quotes' in TeX, and the usage difference between a hyphen (-), en-dash (--) and em-dash (---).}
%
%It is very advisable to skim through a book about scientific English writing before starting the thesis. I can recommend `\citetitle{glasman2010science}' by \citet{glasman2010science}.

\chapter{Theory}

% TODO: what the theory is about
% TODO: photogrammetry vs lidar

\section{Feature-based photogrammetry}
Broadly speaking, photogrammetry is the process of obtaining information about objects (namely their model) using images.
Because the problem is difficult and involves a lot of steps and specialized algorithms, many open-source and proprietary programs have been developed to simplify this task.

The thesis uses the Agisoft Metashape software due to its highly configurable API.
Since it is proprietary and closed-source, there is no easy way to describe and cite the exact algorithms used.
However, a forum statement from the lead developer Dmitry Semyonov \parencite{metashapeForumPost} outlined the general methods used, which this section aims to cover.

Starting with a set of images and ending with a textured model, the process can be split into the following parts:

\begin{enumerate}
	\item \textbf{feature extraction:} for each image, find features that are stable under linear and affine transformations, 3D viewpoint change and illumination
	\item \textbf{feature matching:} match the features across multiple images
	\item \textbf{bundle adjustment:} find the approximate positions of the camera and the matched features (obtaining „structure from motion“ data)
	\item \textbf{dense pointcloud:} generate additional features for higher model quality
	\item \textbf{surface reconstruction:} construct the surface of the model
	\item \textbf{texturing:} apply texture on the constructed model
\end{enumerate}

The following sections aim to cover each of these in part.

\subsection{Feature extraction}
The image features are extracted using the Scale Invariant Feature Transform (SIFT) method described in \citet{lowe1999object,lowe2004distinctive,snavely2008modeling}.
Each of the features is a vector, invariant or partially invariant to linear and affine transformations, illuminance change and 3D transformations.
To find such vectors, a Gaussian scale-space is used.

Formally, a scale-space for a given 2D image $I(x, y)$ is defined as a function
$$L(x, y, \sigma) = G(x, y, \sigma) \times I(x, y)$$

The $\times$ symbol is a convolution of the two functions in $x$ and $y$, and $G$ is the Gaussian function
$$G(x, y, \sigma) = \frac{1}{2\pi \sigma^2} e^{-(x^2 + y^2) / 2\sigma^2}$$

This operation is then applied twice on image $I$ with $\sigma = \sqrt{2}$, yielding images $I_{1,1}, I_{2,1}$, the difference of which yields the image $G_1$.
After this, the image $I_{2,1}$ is downscaled by a certain factor (the original method uses $1.5$ with bilinear interpolation) and the process is repeated, yielding $I_{1,2}, I_{2,2}$ and $G_2$, respectively.

\begin{figure}
	\centering
	\includegraphics[width=\columnwidth]{images/gauss.png}
	\caption{An example of the convolution and difference applied to a hold image.}
\end{figure}

To find the features, the local minima and maxima (compared to the 8 neighbouring pixels) of $G_1$ are examined -- if they are also minima/maxima in $G_2$ (accounting for the downscaling), they are chosen as the key points.

The intuitive idea behind this approach is to suppress the finer details of the image, since each of the features in the coarse version of the image should be present in more detail in the original \cite{scalespace}.

To calculate the vectors for each of these key points, the magnitude $M(x,y)$ and orientation $R(x,y)$ is calculated using their neighbouring points via the following formula:
$$
\begin{aligned}
	M(x,y) &= \sqrt{\left(I_{1,1}(x, y) - I_{1,1}(x + 1, y)\right)^2 + \left(I_{1,1}(x,y) - I_{1,1}(x, y + 1)\right)^2} \\[0.7em]
	R(x,y) &= \mathrm{atan2} \left(I_{1,1}(x, y) - I_{1,1}(x + 1, y), I_{1,1}(x,y) - I_{1,1}(x, y + 1)\right)
\end{aligned}
$$

For the vectors to be invariant to orientation and contrast changes, a canonical orientation is calculated from their local image gradients (again using the Gaussian function with some parameter $\sigma$).

\subsection{Feature matching}

% SIFT still likely used here?
% These descriptors are used later to detect correspondences across the photos. This is similar to the well known SIFT approach, but uses different algorithms for a little bit higher alignment quality.

\subsection{Bundle adjustment}
After the features have been determined and matched across the images, the next step is to simultaneously calculate their position in the 3D space and also the positions of the cameras.
This is referred to as the bundle adjustment problem \cite{snavely2008modeling,schneider19913}, the name referring to bundles of rays from the predicted positions of the points going into the cameras.

Formally, we can model the scene by a vector of 3D points $\mathbf{X}_{i \in [n]}$, taken from cameras with parameters (position, focal length, etc.) given by the vector $\mathbf{P}_{j \in [m]}$.
Our matched features are observations $\underline{\bm{x}}_{ij}$ of point $i$ from camera $j$.

Let's now assume that we have a function $\bm{x}(\mathbf{X}_i, \mathbf{P}_j)$ that takes the feature position $\mathbf{X}_i$ and camera position $\mathbf{P}_j$, and returns the observation position $\mathbf{x}_{i, j}$.
Using this, the bundle adjustment problem can be formulated as a minimization problem of the function $$\Delta x_{i, j} (\mathbf{X}_i, \mathbf{M}_j) = \underline{\bm{x}}_{ij} - \bm{x}(\mathbf{X}_i, \mathbf{P}_j)$$
given the observed features and an appropriate cost function.

% TODO: non-linear least squares here

\subsection{Dense pointcloud}
To generate a high quality model, additional points have to be matched.
This is 

\subsection{Surface reconstruction}
% TODO: přečíst poissona
% \cite{kazhdan2006poisson}

\subsection{Texturing}
% TODO: ?

\section{Using markers}
In certain cases (such as this one), it is necessary for the generated model to be correctly scaled and positioned (in either local or world coordinates) such that it corresponds to its size in the real world.

A simple approach for local coordinates is to use easily recognizable markers placed around the scanned object.
Since their position in space is known (measured in advance), it can be used to infer the correct scale of the entire model.

\begin{figure}
	\centering
	\includesvg[width=0.7\columnwidth]{images/targets.svg}
	\caption{18th 14-bit marker (left) and the corresponding binary value (right). The grey segments denote the opposing pair of $1$ bits and the blue circle denotes the parity bit.}
\end{figure}

The type of the markers used in Metashape is based on \citet{schneider19913,borisPatent}.
Each circular target contains an inner and outer circle, with the inner serving the purpose of locating the marker in the images and the outer for encoding the marker data.

The exact number of stored bits depends on the size of the marker, most common beind $12$ and $14$ respectively.
The $0$ bits correspond to white segments and $1$ to black ones.
For decoding correctness and robustness, the following constraints are additionally imposed on each of the targets:

\begin{itemize}
	\item the targets must be rotationally invariant
	\item the first bit is a parity bit ($0$ for odd, $1$ for even)
	\item there must exist an opposing pair of $1$ bits
\end{itemize}

While this does reduce the number of viable targets (see \cref{tab:markers}), it is still more than enough for the purposes of scanning smaller objects (and thus for the porposes this paper).

\begin{table}
\centering\footnotesize\sf
\begin{tabular}{rrr}
\toprule
Bits & All targets & Valid targets \\
\midrule
12 & 4096   & 147 \\
14 & 16384  & 516 \\
16 & 65536  & 1861 \\
18 & 262144 & 6766 \\
\bottomrule
\end{tabular}
\caption{The number of targets, depending on the desired number of bits \cite{targetsPost}.}
\label{tab:markers}
\end{table}

\chapter{Realization}

The realization of the project can be separated into three distinct parts: \textbf{creating wall model}, \textbf{creating hold models} and the \textbf{editor implementation}.
Each of the respective parts are open-source projects on GitHub, licensed under GPLv3:
\begin{itemize}
	\item \raisebox{-0.08em}{\includesvg[height=0.85 \baselineskip]{images/clis.svg}} -- the climber's scanner \footnote{\url{https://github.com/Climber-Apps/Clis}}
	\item \raisebox{-0.08em}{\includesvg[height=0.85 \baselineskip]{images/cled.svg}} -- the climber's editor \footnote{\url{https://github.com/Climber-Apps/Cled}}
\end{itemize}

\section{Creating wall model}
The wall model (figure \ref{fig:model}) has been created semi-automatically.
First, a set of 188 images was used by the Agisoft Metashape software to create a reference model.
This model was then manually edited in Blender 3D to remove possible modelling errors.
Since the wall consists only of straight segments connected together, this can be done relatively easily.
After manual changes, the resulting model was reimported to Metashape to generate the texture from the photos.

\begin{figure}[H]
	\centering
	\includegraphics[width=\columnwidth]{images/wall/image.png}
	\caption{The model of the Smíchoff wall, created using Metashape and Blender.}
	\label{fig:model}
\end{figure}


\section{Creating hold models}
Since a regular climbing wall contains hundreds or even thousands of holds of varying sizes, it would be infeasible to model each of them manually.
It is important to automize as many steps as possible so that the amount of manual work done is minimized.

This has been acomplished using a turntable-based workflow, upon which the holds are placed and scanned.
The turntable turns the hold around for a stationary camera to take photos, which are then fed into the Agisoft Metashape protogrammetry software (+ Blender for post-processing) to generate the model.

The resulting workflow can process a single hold in $\sim 40$ seconds of scanning (taking $12$ photos and varying depending on the lighting conditions), plus  $\sim4$ minutes of processing.

\subsection{Turntable design}
A three stepper-motor turntable has been developed and 3D printed using the Fusion 360 CAD software and an Ender Pro 5 3D printer (figure \ref{fig:turntable}).
An Arduino board and A4988 motor controllers are used for fine motor control (figure \ref{fig:wiring}).
The communication with the Arduino is implemented via the serial port.

\begin{figure}
	\centering
	\includesvg[width=\columnwidth]{images/wiring.svg}
	\caption{The wiring diagram of the turntable, created using Fritzing.}
	\label{fig:wiring}
\end{figure}

To minimize the amount of friction, the only points of contact between the top and bottom part are 7 bearings (6 placed in a circular pattern and 1 directly in the middle).
Additionally, gears mounted to the motors turn the top of the turntable, achieving a TODO reduction.
This allows the turntable to handle objects of weight up to TODO (heavier objects have to be turned manually).

\begin{figure}
	\centering
	\subfloat[\centering Top part.]{{\includegraphics[height=3.6cm]{images/turntable/top.png} }}%
	\hfill
	\subfloat[\centering Base part from the top.]{{\includegraphics[height=3.6cm]{images/turntable/bottom.png} }}%
	\hfill
	\subfloat[\centering Base part from the side.]{{\includegraphics[height=3.6cm]{images/turntable/side.png} }}%
	\caption{Parts of the model of the turntable, created using Fusion360.}%
	\label{fig:turntable}
\end{figure}

TODO: the motor mount models and the motor wheel model

The top area is connected to a plexiglass with a marked center and 4 markers.

\subsection{Environment setup}
Two LED lights, mounted on adjustable stands and covered with baking paper (for light diffusion) were used to achieve good lighting conditions.
A large black cloth was used as a background around the turntable to minimize the number of unwanted image features (see figure \ref{fig:setup}).

Additionally, since the camera is static, 300 ISO and 22 f/s were used to maximize the image quality.
This meant that the exposure time was usually around $~1.5$ seconds.

The entire setup can be carried in a backpacking pack\footnote{Excluding the plexiglass, which is inflexible, and the camera, which is fragile.}, making it portable.
Additionaly, all of the setup components (including the turntable, excluding the camera) can be purchased for under $4000$ CZK, making it relatively affordable.

\begin{figure}
	\centering
	\includegraphics[width=\columnwidth]{images/setup/setup.png}
	\caption{An image of the setup during scanning, including the taken image. f-stop: 22, focal length: 38mm, ISO: 320, exposure time: 1s.}
	  
	\label{fig:setup}
\end{figure}

\subsection{Dual texture holds}
Glossy surfaces are one of the most problematic types surfaces for photogrammetry, since the angle under which they are imaged change their appearance due to light reflecting into the camera.
This poses a problem for „dual-texture“ holds which, as the name suggests, contain two textures -- a mate texture that is meant for the climber to hold (or step) onto, and the glossy texture which is usually not.
They are increasingly used in modern bouldering and sports climbing, because they give setters more freedom in creating difficult routes.

A standard way of dealing with glossy surfaces is to cover them in something that is mate.
While this does solve the problem of model generation, it ruins the texture, because the mate solution is usually opaque (and possibly requiring another set of images for texturing).

In our case, however, the solution is rather obvious -- cover them in climbing chalk (figure \ref{fig:chalk}).
Since it is mate, it reduces the reflections of the glossy surface and provides additional feature points, making it easier for the photogrammetry software to reconstruct the model.
Additionally, since climbing chalk will be applied to the holds by climber during regular usage anyway, there is no need to obtain more images for texturing.

\begin{figure}
	\centering
	\subfloat{{\includegraphics[height=2.5cm]{images/holds/1.png} }}%
	\hfill
	\subfloat{{\includegraphics[height=2.5cm]{images/holds/2.png} }}%
	\hfill
	\subfloat{{\includegraphics[height=2.5cm]{images/holds/3.png} }}%
	\caption{Example of holds with chalk applied. Some reflections are still visible, but they are not as pronounced as with no chalk applied.}%
	\label{fig:chalk}
\end{figure}

\section{Editor implementation}
The editor has been created using the Unity editor, with the majority of the codebase being written in C\#.
Besides the builtin libraries, a number of freely available user packages were used for simplifying the coding process, namely

\begin{itemize}
	\item \textbf{Quick Outline} \footnote{\url{https://assetstore.unity.com/packages/tools/particles-effects/quick-outline-115488}} -- object outline creator (for hold highlights).
	\item \textbf{OBJImport} \footnote{\url{https://assetstore.unity.com/packages/tools/modeling/runtime-obj-importer-49547}} -- an \verb|.obj| file importer (including texture files).
	\item \textbf{StandaloneFileBrowser} \footnote{\url{https://github.com/gkngkc/UnityStandaloneFileBrowser}} -- a cross-platform file browser.
\end{itemize}

The editor is cross-platform and can be run on Mac, Windows and Linux, with the newest builds being freely available at the project GitHub page.

\subsection{Modes}
The editor contains three modes in which can operate, with the current mode being always displayed in the top right.
While this might seem like an implementation detail, it is arguably important for user documentation and general usage.
An obvious inspiration is the Vim text editor, which makes the mode distinction similarly apparent.
The modes are the following:

\begin{itemize}
		\item \textbf{NORMAL} -- the mode the user is usually in; hovering highlights holds, which can be either picked up or selected.
		\item \textbf{HOLDING} -- the mode in which the user is holding a specific hold.
		\item \textbf{ROUTE} -- the mode in which the user modifies the selected route.
\end{itemize}

\subsection{Hold picking}
Picking which holds to place on the wall quickly is crucial for effective virtual route setting, because it dictates the time the route setting takes.
The editor contains a hold picking menu from which a subset of holds can be selected and cycled while editing.
Filtering holds can be done by selecting a specific hold color, type, manufacturer, or custom hold label.
Additionally, hovering on each of the holds rotates them around, which is useful when the static hold image isn't descriptive enough.

TODO: obrázek toho vybírání


\subsection{Import and Export}
The import and export format for the project files is a human-readable \verb|yaml| file containing information about the state of the holds, routes, paths to the models and other metadata.
This makes it easy to be used in other applications (both for viewing and modification), since it is well readable and parsable.
Here is an example of one such file:

TODO: příklad souboru

\subsection{Capturing images}
The editor includes functionality for capturing the current image of the wall.


\chapwithtoc{Conclusion}
Clis and Cled have been implemented and tested on the Smíchoff bouldering wall.


TODO: repeat what I achieved

TODO: practical applcations

Future developments mainly include implementing a web interface so the exported routes can be interacted with by regular users.
Such interface should support the ideas outlined in the introduction, such as route filtering, liking/disliking routes and send videos.

Additional features that were beyond the scope of this project but should also be implemented in the future include automatically selecting holds from a photo of the wall, „snapping to hole“ for holds that can only be placed in specific ways, and a large number of quality-of-life improvements and bugfixes.

\include{bibliography}

\appendix
\chapter{Clis quickstart}\label{apx:clis}

\section{Setting up}\label{apx:clissetup}

For the setup, you will need to first install \mintinline{text}{pyenv}:
\begin{minted}{text}
# make sure to read what the script does before running it!
curl https://pyenv.run | bash
\end{minted}
and configure your shell's environment: \url{https://github.com/pyenv/pyenv\#basic-github-checkout}.
You will then need to install Python \mintinline{text}{3.7.x} (due to Metashape and Blender working only with older Python versions):

\begin{minted}{text}
pyenv install -v 3.7.12
\end{minted}

To set up the virtual environment, do (in this directory):

\begin{minted}{text}
pyenv virtualenv 3.7.12 clis
pyenv local clis
pyenv activate clis
pyenv exec pip install -r requirements.txt
pyenv exec bpy_post_install # locates Blender and copies over
                            # some of its scripts
\end{minted}

\section{Contents}

Before running the scripts (using \mintinline{text}{pyenv exec}), it is advisable
to check the configuration file \mintinline{text}{config.py}, since a~lot of the values will most likely
differ from their default values for your particular setup. Each of the
respective folders contain a~\mintinline{text}{README.md} that further explains
the usage of each of the scripts.

\begin{itemize}
	\item \mintinline{text}{01-scanning/} --- tools for scanning the holds.
	\item \mintinline{text}{02-processing/} --- tools for processing the hold images into models.
	\item \mintinline{text}{03-data/} --- hold data format specification + tools for managing the models.
\end{itemize}

\section{Usage}

\subsection{Using tasks}

To simplify the usage of Clis, various pre-programmed scripts have been added.
To create a~single model using them, you can do the following:

\begin{itemize}
	\item \mintinline{text}{./scan_single.sh} to scan a~single hold,
	\item \mintinline{text}{./move_from_camera.sh} to move the hold photos from the camera,
	\item \mintinline{text}{./convert_from_raw.sh} to convert the photos to a~usable format,
	\item \mintinline{text}{./generate_models.sh} to generate the model and finally
	\item \mintinline{text}{./add_models.sh} to add the model to the \mintinline{text}{holds.yaml} file.
\end{itemize}

\subsection{Using scripts}

Run

\begin{minted}{text}
pyenv exec python 01-scanning/01-scan.py automatic 15 \
	&& pyenv exec python 01-scanning/02-copy.py camera \
	&& pyenv exec python 01-scanning/03-convert.py
\end{minted}
to automatically take 15 pictures of the hold, copy them from the camera
and convert them from raw. If you want to take them manually because you
don't have the turntable, replace \texttt{automatic} with
\texttt{manual}. Then run
\begin{minted}{text}
pyenv exec python 02-processing/01-model.py
\end{minted}
to generate the model. Finally, run
\begin{minted}{text}
pyenv exec python 03-data/01-add_models.py
\end{minted}
to add the information about the hold to the \texttt{holds.yaml}
dictionary.

\chapter{Cled quickstart}

\begin{table}[H]
\begin{tabular}{@{}ll@{}}
\toprule
\multicolumn{1}{c}{\textbf{Key}}     & \multicolumn{1}{c}{\textbf{Action}}           \\ \midrule
left click                           & pick up/place the hovered/held hold           \\
right click                          & select the hovered route                      \\
                                     &                                               \\
escape                               & opens the escape menu                         \\
e                                    & toggles between \textbf{NORMAL} and \textbf{EDITING}\\
tab or q                             & opens holds menu                              \\
                                     &                                               \\
d or delete                          & delete the hovered hold                       \\
shift + left press + mouse (\textbf{EDITING}) & rotate the held hold                          \\
t/b                                  & toggle the hovered hold as ending/starting    \\
control + left click (\textbf{ROUTE})         & toggle the hovered hold in the route \\
wheel up/down                        & cycle the selected holds                      \\ \bottomrule
\end{tabular}
\label{The editor key bindings.}
\end{table}



\chapter{Clis source code}

\begin{forest}
  pic dir tree,
  where level=0{}{% folder icons by default; override using file for file icons
    directory,
  },
  [Clis
    [01-scanning, label=right:Utilities for scanning the objects.
      [markers, label=right:Marker models and images.]
      [turntable, label=right:Turntable models and code.]
      [tutorials, label=right:Tutorials on scanning both the holds and the wall.]
    ]
    [02-processing, label=right:Utilities for generating models from scans.]
    [03-data, label=right:Utilities for working with the generated models.]
    [tasks, label=right:Pre-programmed tasks.]
    [config.py, file, label=right:Clis configuration.]
    [utilities.py, file, label=right:Various utility functions used in Clis.]
  ]
\end{forest}

\chapter{Cled source code}

\begin{forest}
  pic dir tree,
  where level=0{}{% folder icons by default; override using file for file icons
    directory,
  },
  [Cled
    [Assets
      [Materials, label=right:Materials for starting and ending markers.]
      [Prefabs, label=right:Starting and ending marker prefabs.]
      [Scenes, label=right:Game scene and lighting settings.]
      [Script, label=right:Editor code.]
      [UI, label=right:UI semantic and styling files.]
      [tutorials, label=right:Tutorials on scanning both the holds and the wall.]
    ]
    [Models, label=right:Sample wall and hold models.]
    [Scripts, label=right:Scripts for importing data from Clis.]
    [CHANGELOG.md, file, label=right:Contains the list of changes for each version.]
  ]
\end{forest}

\chapter{Cled release 2.3.1}

\begin{forest}
  pic dir tree,
  where level=0{}{% folder icons by default; override using file for file icons
    directory,
  },
  [Cled
    [Windows, label=right:Windows release executable files.]
    [Linux, label=right:Linux release executable files.]
  ]
\end{forest}


\openright
\end{document}
